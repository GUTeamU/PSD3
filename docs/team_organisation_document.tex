\documentclass{article}

\title{Team Organisation Document\\Team U}

\author{
    Michael Cromie\\
    \and
    Fraser Leishman\\
    \and
    Edvin Malinovskis\\
    \and
    Andrew McDonald\\
    \and
    Matthew Paterson
}

\begin{document}
\date{\today}
\maketitle

\section{Proposed Team Organisation Model}
The team is proposing to use the SCRUM approach which is a part of the Agile
software development framework.   The use of the scrum method is suitable as it
is the most flexible of the options. Having this flexibility is really useful
as it allows each team member to contribute to different parts of the
development process and work towards their strengths.


The incremental development methods associated with agile methods are also
essential due to the importance of performing unit tests on each increment
before they can be integrated with the rest of the system, to ensure the
production of a high quality end result.


The agile model is particularly attractive as it allows all members to work
towards a common goal, rather than being tied down to traditional sequential
methods. Verbal communication between all team members is a must in order to
maintain organisation and ensure members are informed and aware of what the
others are doing. Dealing with changing requirements is another important
aspect of the model, as the team must be able to efficiently adapt to any
emerging needs of the client. 

\section{Description of the Roles within the Team}
\subsection{Scrum Master}
A scrum master will arrange and lead regular group meetings and discussions.
He/she will discuss with the team at these sessions daily progress and any
problems encountered during recent tasks. If problems are discovered, the scrum
master will find the appropriate tools or personnel among the team to overcome
these difficulties.
\subsection{Developers}
A developer within a team will contribute to the identification, analysis and
implementation of each task assigned to the project. This will allow the
project to be completed efficiently and effectively, resulting in high quality
software products being produced. 
\subsection{Client Representative}
A client representative will be embeded into the development team in order to 
ensure that client's requirements are met and to help answer any questions that 
the developers might have regarding the project outside of the scheduled meetings
with the client.

\section{Provisional Assignment of Team Members to Role}
\subsection{Fraser Leishman --- Scrum Master}
Fraser has good organisational skills and excellent communicational skills
through his part time work, dealing with the public, in John Lewis. There he
has learned to work well as part of a team and can apply these skills to any
team based development projects he will face in the future.

\subsection{Michael Cromie --- Developer}
Michael has a firm grasp of software development techniques due to his work
experience at the tech company Atos, where he observed experienced programmers
as they worked on professional software for the NHS. This will stand him in
very good stead to go on and aid the team in developing high quality software,
as he knows how a professional software house operates on a daily basis, which
he can bring to the group.

\subsection{Andrew McDonald --- Developer}
Due to his first two years studying electrical engineering at the University of
Glasgow, Andrew is very technically minded. He has a very hands on approach,
which when brought to the team will help solve any technicalities faced on the
development path. Andrew will also be a key member in projects involving
electronics or robotics.

\subsection{Matthew Paterson --- Developer}
Matthew is a problem solver. He has strong mathematical interests along with a
passion for algorithms, which will allow him to furtherly break down and
analyse tasks. This theoretical approach to the development process will allow
the team to foresee potential miscalculations and avoid them in due course. 

\subsection{Edvin Malinovskis --- Developer}
Edvin is an experienced developer and a key member of the team. He can program
in several languages and has had experience as a freelance developer,
completing numerous real-world software products. Edvin has a very software
based mindset which will make him central to the development of high quality
software. 

\section{Communication Strategies}
Our main strategy for project organisation is to use Asana, a web and mobile
application designed to improve the way teams communicate and collaborate, as a
metaphorical virtual whiteboard we can rally around. We compared other project
management software, such as Trello, however we settled on Asana as we
preferred the user interface.

We have created a Facebook group and are utilising the Google Documents
facility to allow the team to connect and cooperate when a live meeting is not
available.

We will also be using the version control software Git in order to share and
coordinate the production of our software solution. This will be hosted on the
University allocated server space once it becomes available; whilst also being
backed up to a GitHub repository.

\section{Organisational Risks}
If someone was to leave the team, the other four members would carry on with
any on going projects. The other members would take over the leavers role in
the team, meaning the task could go on. This would be possible due to the agile
organisation model and thorough communication strategies, which means all team
members can be flexible in their roles and well informed as to the progress
being made on other tasks.

If, at any point, there is an argument between team members on certain issues,
such as how to approach tasks, we are likely going to try and solve the issue
within the group and get the views of the other members and vote democratically
on who has the better idea/correct vision of what is to be done. Potentially, a
compromise could also be available which would hopefully appease the members of
the group who were arguing.

However, as a group, we could also bring the issue to the attention our
supervisor, he would give impartial advice on the issue which we would then
take away to debate amongst ourselves.
\end{document}
